%!TEX program = xelatex
% 完整编译方法 1 pdflatex -> bibtex -> pdflatex -> pdflatex
% 完整编译方法 2: xelatex -> bibtex -> xelatex -> xelatex
\documentclass[lang=cn,11pt]{elegantpaper}

\title{设计报告}
\author{\rightline{南开大学1队}  \\ \rightline{朱彦谕、王继铭、刘炼}}

% \institute{\href{https://elegantlatex.org/}{Elegant\LaTeX{} 项目组}}

% 不需要版本信息,直接注释即可
% \version{1.0}
% 不需要时间信息的话,需要把 \today 删除。
\date{}


% 如果想修改参考文献样式,请把这行注释掉
\usepackage[authoryear]{gbt7714}  % 国标

\begin{document}

\maketitle

\section{“爱”的发现}
在宗教与艺术中,都对人与人之间的“爱”做出了形象而深刻的探讨。这里以剧本《俄狄浦斯王》与基督教经典《圣经》为例,简要介绍其中“爱”的含义。
\subsection{艺术中的“爱”——以《俄狄浦斯王》为例}
《俄狄浦斯王》作为一部“十全十美的悲剧”[1],其揭示了伟大英雄自身命运中不可避免的悲剧性,然而,它也表现了一种非凡的超越性。在其中,有多处关于“爱”的体现。
\subsubsection{伊俄卡斯特之“爱”}
作为俄狄浦斯的生母,伊俄卡斯特是一个典型的“母亲”形象,人性的爱与物性的爱在她的身上交织冲突。她害怕面对真相,然而却又无法逃脱悲剧的命运,她对于事实早已明晰,然而对儿子和丈夫的爱,使得她逃避,最后自杀。
她是平凡的人的化身,是我们的真实写照。人性中的“爱”的怯弱,“爱”的自私,使得她屈从于自己的命运与罪孽。
\subsubsection{俄狄浦斯之“爱”}
俄狄浦斯的“爱”,是对于城邦的“爱”,是超越的人类的“爱”,他的在自我追寻、自我探求中诞生的悲剧[2],使得他超越精神上与命运上的局限。“人类真正的伟大之处恰恰寓于他谜一样的本
质———疑问———之中”[3]超越性的“爱”,在此中真正得以诞生。
\subsection{宗教中的“爱”——以《圣经》为例}
《新约》中言:“爱是恒久忍耐,又有恩慈;爱是不嫉妒,爱是不自夸,不张狂,不作害羞的事,不求自己的益处,不轻易发怒,不计算人的恶,不喜欢不义,只喜欢真理;凡事包容,凡事相信,凡事盼望,凡事忍耐;爱是永不止息。”[4]
《圣经》中有一故事:众人要用石子砸死一妓女,耶稣说:“你们当中有谁是纯洁无暇的,就站出来向 她投一块石头吧!”因而,基督教中的“爱”,是救赎,是仁慈,是洗清身上的罪孽。这样一种赎罪的心理,是使得人之为人,实现康德意义上的道德的无可避免的道路。

\section{以“爱”之名实现哲学}
哲学的“何以为”的问题,实际上是人何以为人的问题,即如何突破人的物性(兽性),达到所说的“精神青春期”[5]。

\subsection{“爱”与“美”}
“爱”在于儒家所说“泛爱众”,也在于“爱一切美”。对于美的欣赏在于对于“爱”的正确认识。爱是以“见证人”的角度去观察人与世界的一切情绪,从而达到康德所说的人先天意义上的共通感[6]。所谓“美”,更大的意义上,是有审美能力的人所界定的,这种审美能力来源于人的共通,而这种共通的发展,来自于对于自身的超越。如俄狄浦斯,克服了对悲剧命运本身的恐惧,而实现了对生命的超越,达到了新的“泛爱”的高度,才能在更高层次的意义上谈审美。
即本质上,“爱”是“美”发展的基础。

\subsection{“爱”与“善”}
柏拉图《大希庇阿斯篇》在论证美的过程中,得出的结果是“美是难的”[7],但更应该值得注意的是,其提到美即是善,虽然在探索中发现这是错误的,但这样的探讨是有意义的。善与美不是相等的关系,但它们都以“爱”作为根基。
所谓的善,来自于一种亚里士多德诗学中所定义的“懊悔”心理[8],而这种懊悔,来自于对于悲剧作品的自身感悟,即人之“爱”引发的慈悲之心。这样一种慈悲,和佛教中所言及的慈悲极大相似,有着十分重要的意义。
这种慈悲,同样来自于“爱”,其超越人的自私与恐惧,而形成一种有意义,有形式的“大爱”。爱与行善是同根的,亦是同一的;对于“善”的实现,即是通过对“爱”的启发而来。

\subsection{“爱”的哲学实现}
哲学的实现,在于道德与美的判断的实现。通过对于“爱”与“善”和“美”之间关系的阐发,完全可以明晰,对于哲学上的实现,实际上是对于人自身的实现而言的;即人本身的自我超越,其来自于对于原罪的审视以及对于人自身悲剧命运的超越。何以摆脱人生来的兽性,而实现人自身的完全自由,是哲学实现的基础,也是宗教和艺术一直在追寻的事物。

\begin{thebibliography}{1}

  \bibitem{fu}[1]守祥. 《俄狄浦斯王》:命运主题与悲剧精神的现代性[J]. 世界文学评论, 2006(1):232-237.
  \bibitem{xu}[2]许纾. 悲剧的命运形式新探——重读经典《俄狄浦斯王》[J]. 河南师范大学学报(哲学社会科学版), 2006, 33(5):143-145.
  \bibitem{chen et al.}[3]陈洪文, 水见馥.古希腊三大悲剧家研究[M] .北京:中国社会科学出版社, 1986
  \bibitem{wei et al.}[4]魏玉奇, 李娟. 圣经新约名篇精选:英汉对照[M]. 天津人民出版社, 1998.
  \bibitem{zhu}[5]朱鲁子. 人的宣言———人,要认识你自己[M].京: 清华大学出版社、北京交通大学出版社,2007.
  \bibitem{kant}[6]康德. 判断力批判.上[M]. 商务印书馆, 1964.
  \bibitem{Plato}[7]柏拉图. 文艺对话集[M]. 9.
  \bibitem{Aristotle}[8]亚理斯多德. 诗学 诗艺[M]. 人民文学出版社, 1988.
\end{thebibliography}
    
  

\end{document}
