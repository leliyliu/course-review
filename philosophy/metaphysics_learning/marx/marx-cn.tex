%!TEX program = xelatex
% 完整编译方法 1 pdflatex -> bibtex -> pdflatex -> pdflatex
% 完整编译方法 2: xelatex -> bibtex -> xelatex -> xelatex
\documentclass[lang=cn,11pt]{elegantpaper}

\title{我的形而上学之思 \\[2ex]\begin{large} ——对时空观统一与二分的探讨 \end{large}}
\author{刘炼}

% \institute{\href{https://elegantlatex.org/}{Elegant\LaTeX{} 项目组}}

% 不需要版本信息,直接注释即可
% \version{1.0}
% 不需要时间信息的话,需要把 \today 删除。
\date{ \\ 时间:\today \\ 学号: 1711361}


% 如果想修改参考文献样式,请把这行注释掉
\usepackage[authoryear]{gbt7714}  % 国标

\begin{document}

\maketitle

\begin{abstract}
\noindent 关于时空的探讨是形而上学中一个重要的问题,但是,不论是基于先天理性直观的方式还是基于经验的方式来对于时空进行探讨,都有一定的局限性。同时,传统的关于时空的探讨都存在时间和空间进行二分的问题,这其中的正确性使值得探讨的。实现理智直观和感性经验上的时空观的统一性理解,超越传统的时空二分的观念,对于形而上学新的体系的构建,是必要的,也是必须的。
\keywords{先天理性;感性经验;传统时空观;二分与统一;超越;形而上学}
\end{abstract}
形而上学中对于时空的考察一直是一个引人注目的问题。 在康德之前,认识史上各派时空观基本可归结为两大类;一类将时空绝对化,实体化,而另一类则将时空相对化,观念化。[1]而康德在这种对立之中,把认识的对象同认识本身区分开来,将“物自身”和“现象”隔离开来,否定时空在客观世界中的存在,又保留其在主体的认识形式中。

然而,对于时空观的考察到康德之后便难以深入进行,这很大程度上是因为康德中现象与物自身的划分,使得我们无法再探查到理智直观世界中的“时空的样子”,因而,对于时空的考察便局限于非本质的现象之中,这是难以在形而上学的框架下得到实现的。与此同时,传统框架下的时空考察都基于了对时间和空间的分别考察,然而这种划分的正确性也是值得质疑的。

\section{对于时空观的经验与直观}
  要对时空观念进行考察,必须从感性经验与理智直观的二分入手,讨论我们所经验的时空与理智直观的时空的本性是否存在差异,以及这种差异是否是难以跨越的鸿沟。同时,对于时空观念的考察,不得不以康德所提出的时空观念为理解和比较的对象,并通过此种探析,实现对于时空观念新的认识。

\subsection{现象与物自身}
要对时空观进行深入的探查,首先要理解康德对于时空划分中的基础概念——现象与物自身的概念。有人将康德的现象和物自身概念视作两个不同的实体,因而认定他为不可知论者和二元论者,然而这样一种批评是错误的,没有对康德进行深入考究的。

康德对其的现象和物自身的划分进行了具体的说明;“但如果这个批判没有弄错的话,它就在这里教我们从两种不同的意义来设想对象,也就是或者设想为现象,或者设想为自在之物本身”[2](P21)。可以看到,这样一种对于现象和物自身的划分只是在人的理性能力在认识过程中对事物做出的“两个方面”的理解。[3]


\subsection{对于时空的感性经验}

\subsection{对于时空理智直观的考察}

\section{时空观的二分与统一}

\subsection{时空的探查二分性}

\subsubsection{时空的科学现象表达}

\subsubsection{人理智直观的时空存在}

\subsection{时空维度的丈量}

\subsubsection{空间维度的时间丈量}

\subsubsection{时间维度的空间丈量}

\section{时空的一体性探查}


\begin{thebibliography}{1}

  \bibitem{li}[3]李秋零主编:《康德著作全集》第6 卷,〔北京〕中国人民大学出版社2007 年,第232页
  \bibitem{wang}王德贞. 康德先验时空观的探索思路及其启发意义[J]. 理论导刊, 2006(12):68-69.
  \bibitem{}[德]康德. 纯粹理性批判[M]. 邓晓芒译. 北京: 人民出版社,2004.
  \bibitem{}刘作. 康德的“物自身”概念[J]. 兰州学刊, 2011(12).

\end{thebibliography}
    
  

\end{document}
