%!TEX program = xelatex
% 完整编译方法 1 pdflatex -> bibtex -> pdflatex -> pdflatex
% 完整编译方法 2: xelatex -> bibtex -> xelatex -> xelatex
\documentclass[lang=cn,11pt]{elegantpaper}

\title{我的形而上学之思 \\[2ex]\begin{large} ——对时空观统一与二分的探讨 \end{large}}
\author{刘炼}

% \institute{\href{https://elegantlatex.org/}{Elegant\LaTeX{} 项目组}}

% 不需要版本信息,直接注释即可
% \version{1.0}
% 不需要时间信息的话,需要把 \today 删除。
\date{ \\ 时间:\today \\ 学号: 1711361}


% 如果想修改参考文献样式,请把这行注释掉
\usepackage[authoryear]{gbt7714}  % 国标

\begin{document}

\maketitle

\begin{abstract}
\noindent 关于时空的探讨是形而上学中一个重要的问题,但是,不论是基于先天理性直观的方式还是基于经验的方式来对于时空进行探讨,都有一定的局限性。同时,传统的关于时空的探讨都存在时间和空间进行二分的问题,这其中的正确性使值得探讨的。实现理智直观和感性经验上的时空观的统一性理解,超越传统的时空二分的观念,对于形而上学新的体系的构建,是必要的,也是必须的。
\keywords{先天理性;感性经验;传统时空观;二分与统一;超越;形而上学}
\end{abstract}
形而上学中对于时空的考察一直是一个引人注目的问题。 在康德之前,认识史上各派时空观基本可归结为两大类;一类将时空绝对化,实体化,而另一类则将时空相对化,观念化。[1]而康德在这种对立之中,把认识的对象同认识本身区分开来,将“物自身”和“现象”隔离开来,否定时空在客观世界中的存在,又保留其在主体的认识形式中。

然而,对于时空观的考察到康德之后便难以深入进行,这很大程度上是因为康德中现象与物自身的划分,使得我们无法再探查到理智直观世界中的“时空的样子”,因而,对于时空的考察便局限于非本质的现象之中,这是难以在形而上学的框架下得到实现的。与此同时,传统框架下的时空考察都基于了对时间和空间的分别考察,然而这种划分的正确性也是值得质疑的。

\section{对于时空观的经验与直观}
  要对时空观念进行考察,必须从感性经验与理智直观的二分入手,讨论我们所经验的时空与理智直观的时空的本性是否存在差异,以及这种差异是否是难以跨越的鸿沟。同时,对于时空观念的考察,不得不以康德所提出的时空观念为理解和比较的对象,并通过此种探析,实现对于时空观念新的认识。

\subsection{现象与物自身}
要对时空观进行深入的探查,首先要理解康德对于时空划分中的基础概念——现象与物自身的概念。有人将康德的现象和物自身概念视作两个不同的实体,因而认定他为不可知论者和二元论者,然而这样一种批评是错误的,没有对康德进行深入考究的。

康德对其的现象和物自身的划分进行了具体的说明;“但如果这个批判没有弄错的话,它就在这里教我们从两种不同的意义来设想对象,也就是或者设想为现象,或者设想为自在之物本身”[2](P21)。可以看到,这样一种对于现象和物自身的划分只是在人的理性能力在认识过程中对事物做出的“两个方面”的理解。[3]

康德对于现象和物自身的划分来源于其“先天综合判断”的概念,这一概念超越了经验论和唯理论的争论,然而,其所形成的的知识仍具有局限;人的“先天综合判断”只能以“现象”为认识范围,我们观察到的对象是符合我们的先天综合判断的,而不是本体的“物自身”
在康德那里,只有创造者能够把握物体本体的物自身,在另一个层面上,只有造物主可以通过无限理智直观到物自身。
\subsection{对于时空的感性经验}
在康德看来,我们所能把握到的关于时间和空间的知识,就在我们的感性之内,当我们用我们的先验范畴把感官表象“正确的连贯起来”,就可以获得关于经验现象的客观有效的知识即“真理”[4]。

对于我们所能感性经验到的空间而言,其实质上的经验是对于所框定的空间的界限的划分,是通过质料对空间中的微小的一部分所做出的有限的划分。正如我们对教室这一空间的感性经验,不是来自于空间自身,而是来自于窗户和墙对于空间的框定。

因此,对于时空的感性经验,是人的认识对于人所指向的现象界的把握,其需要符合人的感性经验所构建出的概念系统。故在人对于时空的感性经验下,时空本身只可能是有限的被质料所划定的现象。
\subsection{对于时空理智直观的考察}
虽然人无法拥有像神一样的无限理智,因而无法真正探查到本体的物自身的概念,但康德也对人的理智直观做出了界定,即人拥有有限的理智直观。在康德形而上学中的另一重要概念——想象力,给了人通过理智直观时空的可能。
康德将想象力依照是否具有创造性而划分为:“生产性的想象力”和“再生性的想象力”。在《纯粹理性批判》中,康德将想象力的基本含义规定为人的心灵的直观能力。“想象力是把一个对象甚至当它不在场时也在直观中表象出来的能力。......就想象力就是自发性这一点而言,我有时也把它称之为生产性的想象力,并由此将它区别于再生的想象力。”[2]

就人的再生性的想象力而言,其实质上是可以通过有限理智对于本体的物自身进行有限的直观探查的手段。比如,人在想象力中可以构建一个无界限划分的可能空间;这是从人的认识层面的现象界到物自身界的中间桥梁,然而,其仍然无法到达最终的物自身界。
这很大程度上是由于人的想象力的桥梁性是建立在人的感性经验所认知的现象知识上的。

\section{时空观的二分与统一}
时空的概念在两个层面上仍然具有二分性,对其是否能实现一种统一是一个值得探讨的问题。因此,需要对这两个层面上的二分性进行再次的阐发,同时在另一些思考维度上重新思考时空的概念,来重新考察对于时空概念的界定。

\subsection{时空的探查二分性}
 时空观在一个层面上的二分是来自于现象界和物自身所存在的本体界的二分。本体性的时空,应该是康德意义上的自在之物,然而,现象界的时空,是符合人的认知观念下的有限的约束界定。除此之外,现代科学的发展,带来了对于时空的不同的认识,这样一种认识,在经验和理性的划界上遭遇了困难;也正因如此,它使得这一讨论有了新的内涵。

\subsubsection{人理智直观的时空存在}
在本文的第一节中,已经探讨了人通过再生性的想象力所直观到的时空;然而,我们仍然需要认清楚一个问题,即这样一个直观到的时空观念,是否具有存在层面的意义。

我们首先需要明确,物自身是作为一种普遍的本体性的存在物本身而存在的,而人的现象界认识是作为契合人的认知观念的一种认知上的存在。然而,作为一种桥梁的再生性想象力观念所构建的时空观,它是作为符合人的思维观念而产生的,其突破了人在现象界观察的界限,但却囿于人的有限理智而无法到达无限理性的直观认识的一种非存在性的观念。
尽管在存在性问题上所存在的困难,作为人理智直观的时空观念,仍然具有十分重要的意义。通过人的有限理智所直观到的时空观念,在我们对于时空观念的深入探讨具有极大的参照性。

\subsubsection{时空的科学现象表达}
对于时空观的探查,不只是出现在哲学的研究中,同样也出现在科学性的研究中。然而,在对科学主义下的时空观内容进行表述之前,首先需要探讨一个问题,即科学的观测处于经验还是理性之中?
传统的认识观念认为,科学现象是一种基于经验之下的观测结果,而远非一种理性观念下的非感性化直观。对于传统的观测而言,其确实具有这样的性质,其是对于一般经验现象的观测。
然而,对于爱因斯坦概念上的时空观,其却超越了这一意义。传统的科学性观测,是在实验意义基础上所构建的概念,然而,对于爱因斯坦的时空观,其是在现象之上的一种猜想性的假设,而通过多种实验结果而实现在现象界的验证。
从这样一种意义上而言,时空的这样一种科学性的表述,更可能是基于再生性的想象力。

那么,在此基础上,我们可以对爱因斯坦的时空观的内容进行考察。爱因斯坦的时空观,即时空的存在是一种绝对而统一的存在;然而,空间和时间都是有界限的,并非无限的自在之物。
爱因斯坦的时空观可以带给我们两个方面的考察:1.科学探查到的有界限的时空观念是对人先天认识的现象界限,还是物自身所具有的本体界限? 2.关于时间和空间统一性的考察,时间和空间的统一性表述是否可能?
对于这两个问题的考察,将在下文进行。

\subsection{时空维度的丈量}
关于时空维度的丈量问题,即是在考察时间和空间概念是否可以统一这一问题。我们必须思考这样一个问题,即我们是否可以对时间和空间概念进行丈量,同时,对于时空概念绝对性和自在行的可行性根基进行探查。

\subsubsection{空间维度的时间丈量}
在人的先天认知中,我们所观察到的自然界(或者说现象界)的空间是由外部的质料(例如教室空间是由墙和窗户)界定的,故对于现象界的空间的丈量以及划分,是经由有限的质料在符合我们的先天认知的范畴中实现的。
然而,在人的另一种认识方式,即利用人的想象力所构建的认识世界中,尤其是对于“生产性的想象力”,其中的空间概念的丈量中,除了想象力中所构建的质料,也无法缺少时间这一概念。

例如,在对于上课中的教室的这一空间的想象力构建中,时间观念悄无声息地出现在了丈量这一空间的范畴之中;正如我们对于同一个课堂空间中的想象,虽然整个上课的远非一瞬,然而,想象力所构建的空间却总是由一瞬又一瞬的时间片段所搭建的。

如果我们承认这样想象力的直观方式是对于人的现象界认识到无限理智的认识的一个桥梁的话(正如前文所论述的那样),我们就能轻松地发现这样一个变化,即人的想象力直观出的空间比人经由先天认识结构所经验出的空间的丈量维度,只是多了时间这一层。
\subsubsection{时间维度的空间丈量}
同样的,我们也需要对时间这一维度进行考察,由于我们将大部分的时空观的考察都放在了空间上,而忽略了时间。这很大程度上是由于对于时间的经验性的把握远没有空间那样容易。如果空间的现象界界定是由经验性的物质或者质料所框定的话,那么对于时间的丈量更多的只能是通过质料在现象界中的存在来形成。
正如康德所表述的那样,当人经验一个事物的时候,实际上是将其符合我们的先天认识的范畴传递给人的认知能力,其中时间和空间就是其中的两个维度。即对于时间这一概念的现象界探查,用其它事物来进行框定反而会限制人对其的理解。如果想要更确切地了解它,那么应更多的人的先天认识范畴本身入手。

通过这样一种方式,我们可以看到,对于一个事物而言,其有时间和空间上的认识范畴;然而,更值得被探究的是,是否在另一种认识层面上,时间和空间不只是丈量一个事物的两种方式,其两者本身的相互丈量是否可能。
而已经探讨过,在想象力所构建的认识世界中,空间是可以由时间丈量的,那么,同样的,是否时间也可以由空间丈量,而实现一种在这一认识层面上的统一性呢。

答案是肯定的。不过这样一种丈量的方式,更多意义上,是在“再生性的想象力”的作用下所构建的。利用“再生性的想象力”,人可以构建一个无限的而不具有任何其它质料的空间,然而,不包含其它质料,并不意味不包括时间这一概念。
相反,人的这样一种想象实际上是对时间的想象,而非直接对于空间的想象。

首先,想象出的空间,必然包含一个时间的观念在其中;其次,由于想象的空间是纯粹而无限的,其不被外物所界定,故其是绝对的。绝对的本体自身无法被同等的范畴所丈量(从另一个维度上而言,即人的想象力无法达到真正的无限理智,故这样一种想象的目的不在于空间,而在于时间,空间只是对时间的丈量手段),故在这种情形下的空间不是由时间的绝对丈量的可能,而相反,时间却成为了由绝对的自在之物可以丈量的那个概念。
故出现了这样一种令人欣喜的结果,即,在某一个认识层面上,人可以完成空间与时间的相互丈量,这是之前二分的观点中无法探查到的。
\section{时空的一体性探查}
对于时空的一体性的探查,是一种基于新的科学认识下的对于康德哲学的再审视,其实质上是在探究,时空的一体化是否可能?何以可能?在前面的多个问题的讨论中,我们看到了实现这种一体性的可能,然而,还有一些细节需要说明清楚,才能使得这样一种新的对于时空观的视角得到实际意义的实现。
\subsection{科学探查下的时空观念的再阐发}
基于爱因斯坦时空观的考察以及现代科学的观测和研究方式,我们否决了对于这样一种科学探查下的时空观是完全基于经验的。但是,如果简单地理解为认识只有现象和物自身两个层面,那么不可避免的,这样一种时空观需要被划分到无限理智直观的本体认识中。
显然,这会产生几个矛盾:1.有限的时间和空间的划定无法与本体的超越性,无限性和绝对性相统一  2. 借助人的理智和外在质料的观测是无法触碰到本体性的 3. 维度的区分下的时空范畴发生了交合,破坏了范畴自身的自在性和完整性

因而,转而从想象力的认识范畴进行考察,却恰好可以解决这些问题。想象力的借助经验性和再生性的能力与爱因斯坦对于时空观测中的再发现方式有巧妙的契合——正是通过对于其它事实的经验观测和想象力的再生,使得爱因斯坦突破了原有了牛顿的时空观的范畴。
另一方面,想象力的思维方式为这种有限性指明了方向;同时,其也刚好与在想象力维度下的时空交合的这一表现相契合。因而,完全可以说,这样一种科学探查下的时空观念是在以想象力为基础的认识世界中构建的。
\subsection{时空一体性的实现方式}
对时空一体性这一问题的探查,并没有超出康德所界定的形而上学的概念范畴,然而,在这一种范畴之中,却突破了原有的关于时空的认识观念的考察,在一种新的认识世界的构建基础之上,阐发了时空一体性的可能性。
具体地,对于时空一体性的实现方式,其也没有超出原有的人的认知范畴;相反的,它似乎给人的认识的极限划清了康德所没有划清的边界。如果在康德的概念之中,人的认识性的探查只能基于现象界,而无法触及到本体层面;然而,想象力在时空观念上的这一一体性的突破,使得人的认识能力能够突破一条“河”的距离,而利用这一桥梁,去更多地感知认识世界的另一端的可能。
\begin{thebibliography}{1}

  \bibitem{wang}[1]王德贞. 康德先验时空观的探索思路及其启发意义[J]. 理论导刊, 2006(12):68-69.
  \bibitem{}[2][德]康德. 纯粹理性批判[M]. 邓晓芒译. 北京: 人民出版社,2004.
  \bibitem{}[3]刘作. 康德的“物自身”概念[J]. 兰州学刊, 2011(12).
  \bibitem{}[4]陈晓平. 时间、空间与先验范畴——对康德先验范畴体系的修正[J]. 科学技术哲学研究, 2009, 26(5):6-12.

\end{thebibliography}
    
  

\end{document}
